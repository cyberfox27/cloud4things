\section{Conclusion}
\label{sec:conclusion}
The Internet of Things is a paradigm that revolutionizes the way in that common objects
interact with the environment. In this document we introduce the concept of in the scope of smart
places as a specific IoT application. A smart place is an ecosystem that is composed by smart objects,
in this case RFID tagged objects, that are interconnected within the Internet infrastructure present
in this smart place. However, due to the diversity of smart objects and its different communication
protocols, these smart places are characterized for its heterogeneity which increases the
complexity the execution of management tasks such as deployment and monitoring of the
IoT application that is running in this smart place. The state of the art solutions
reduce the complexity of such tasks by automating the deployment process, but as mentioned
in Section \ref{sec:objectives} this solution requires an high level of expertise in
order to perform these tasks.\\

Therefore, with this work we intend to decrease the complexity of the execution of these
management tasks in two ways. First, we intend to automate the deployment of IoT applications
in smart places by using Cloud orchestration tools, these tools will also allow to describe
the application structure in a high-level perspective, thus modelling the applications
structure only requires to have in mind the application logic. In the other hand, the main
objective of this work is to enable the business owners to define the Service Level Agreements (SLA)
only having in mind the business rules of its smart place. With this approach the business owner will
be able to monitoring the performance of its smart place and establish the amount of resources
needed by the application in order to
have an adequate \textit{QoS}.\\

In the evaluation of the developed solution, we will evaluate the performance of the application
regarding the amount of data that is generated by the objects that are in the smart place.
An important point of the evaluation is to demonstrate the possible scenarios where the smart
place generates more data then the Cloud can process and the opposite scenario where the Cloud is
overpowered. Ultimately, we intend that this work contributes to improving the developed work in
the field of smart places, focusing in an area that is essential to guarantee the correct
operation of them, \textit{QoS} support.
