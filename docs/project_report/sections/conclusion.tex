\section{Conclusion}
\label{sec:conclusion}
The Internet of Things is a paradigm that revolutionizes the way in that common objects
interact with the environment. In this document we introduce the concept of smart
places as a specific IoT application. A smart place is an ecosystem that is composed by smart objects,
in this case RFID tagged objects, that are interconnected to the Internet infrastructure.
However, due to the diversity of smart objects and its different communication
protocols, these smart places are characterized for its heterogeneity which increases the
complexity the execution of management tasks such as deployment and monitoring of the
IoT application that is running in this smart place. The state of the art solutions
reduce the complexity of such tasks by automating the deployment process, but as mentioned
in Section \ref{sec:objectives} these solution require a high level of expertise in
order to perform these tasks.\\

Therefore, with this work we intend to decrease the complexity of the execution of these
management tasks in two ways. The main objective if this work is to automate the deployment
of IoT applications in smart places by using Cloud orchestration tools, that
allow to describe the application structure in a high-level perspective, thus modelling the applications
structure in terms of the application logic. We also intend to guarantee that the Cloud providers
are delivering a service level that meets the expectations. To guarantee that,
Cloud4Things will enable the definition of \textit{QoS} parameters that will be used to
establish Service Level Agreements between the Cloud providers and the customers.\\

In the evaluation of the developed solution, we will evaluate the performance of the
application deployment and the performance of the application that is running in the smart place.
An important point of the evaluation is to demonstrate the possible scenarios where the
smart place generates more data then the Cloud can process and also the opposite scenario where
the Cloud is overpowered. Ultimately, we intend that this work contributes to improving the
developed work in the field of smart places, focusing in \textit{QoS} monitoring that is
essential to guarantee its correct operation.
