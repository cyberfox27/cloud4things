\section{Evaluation Methodology}
\label{sec:evaluation}
The evaluation of the solution will be performed according to two perspectives: one consists in
evaluating the performance of the Cloud and the other consists in evaluate the performance
of the deployment of the application.
% -------------------------------------
%  APPLICATION DEPLOYMENT EVALUATION
% -------------------------------------
\subsection{Application Deployment Evaluation}
\label{sub:application_deployment_evaluation}
The performance of the application deployment is an important aspect to be evaluated.
The deployment operation can be evaluated in terms of the metrics of \textit{Network Bandwith},
\textit{Data Volume} and \textit{Latency} of the process. These values determine
the efficiency of the deployment operation. For instance, if we have a high \textit{Data Volume}
and a low \textit{Network Bandwith} available, the deployment process will have a high
\textit{Latency}.     
% -------------------------------------
%  CLOUD PERFORMANCE EVALUATION
% -------------------------------------
\subsection{Cloud Performance Evaluation}
\label{subs:cloud_performance_evaluation}
An aspect that is important concerns with the performance of the Cloud regarding the amount
of data generated by the events. Cloud computing creates an illusion that available computing
resources on demand are limitless \cite{armbrust2009m}. However, with the increase of the amount of
events, we must measure these computing resources in order to determine if the system is fulfilling
the \textit{QoS} requirements. To perform the evaluation of the behaviour and performance of the Cloud
we need to measure some system metrics such as:
\begin{itemize}
  \item \textit{CPU Utilization:} indicates the percentage of time that the CPU was working at
  the instances in the Cloud. Normally, this metric is available through the Cloud providers.
  Usually, the range of this metrics is given in percentage that can vary between 0-100\%.
  \item \textit{Memory Usage:} indicates the amount of memory that is consumed by the system in a
  given period of time. The unit of this metric is given in MBytes.
  \item \textit{System Load:} is a metric that indicates the general state of the system.
  This metric estimates the general performance of the system by measuring the number of received events.
  The range of this metric varies between 0 and 1. When this metric has a value of 0, it means that the
  system is not receiving any events at the time. If this metric has a value of 1, it means that the system
  is overloaded, consequently the CPU Utilization and Memory Usage metrics are close to the maximum value.
\end{itemize}
