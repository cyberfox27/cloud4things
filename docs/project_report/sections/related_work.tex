\section{Related Work}
\label{sec:related_work}
IoT applications often encapsulate several relatively complex protocols and involves
different software components. Moreover, they require a significant investment in infrastructure,
besides that the system administrators and users spend time with large client and server
installations, setups, or software updates. As most of the computing resources are allocated
on the Internet on servers in Cloud computing, integrating these paradigms in a Cloud-based model
results in a solution with more flexibility of implementation, high scalability and high availability,
and with a reduced upfront investment.\\
% ----------------------------------------
% INTERNET OF THINGS AND CLOUD COMPUTING
% ----------------------------------------
\subsection{Internet of Things and Cloud Computing}
\label{sub:Cloud Computing and Internet of Things}
In RFID-based IoT applications, Guinard et al. \cite{guinard2011cloud} point out that the
deployment of RFID applications are cost-intensive mostly because they involve the
deployment of often rather large and heterogeneous distributed systems. As a consequence,
these systems are often only suitable for big corporations and large implementations and
do not fit the limited resources of small to mid-size businesses and small scale applications
both in terms of required skill-set and costs. To address this problem, Guinard et al. propose
a Cloud-based solution that integrates virtualization technologies and the architecture of
the Web and its services. The case of study consists in a IoT application that uses RFID technology
to substitute existing Electronic Article Surveillance (EAS) technology, such as those used in
clothing stores to track the products. In this scenario they applied the Utility Computing blueprint
to the software stack required by the application using the AWS platform and the EC2 service. To
evaluate the Cloud-based solution, two prototypes was successfully implemented to prove that
the pain points of the RFID applications can be relaxed by adopting the proposed solution.\\

% ----------------------------------
% ENABLING THE CLOUD OF THINGS
% ----------------------------------
Distefano \cite{distefano2012enabling} et al. proposed a high-level modular architecture to implement the
Cloud of Things. According to the authors, things not only can be discovered and aggregated, but also
provided as a service, dynamically, applying the Cloud provisioning model to satisfy the agreed
user requirements and therefore establishing ``Things as a Service'' providers. The ``\textit{Things as a Service}''
(TaaS) paradigm envisages new scenarios and innovative, pervasive, value-added applications,
disclosing the Cloud of Things world to customers and providers as well, thus enabling an open
marketplace of ``things". To address these issues, an ad-hoc infrastructure is required to deal
with the management of sensing an actuation, resources provided by heterogeneous Clouds and
and things. The proposed architecture provides blocks to deal with all the related issues, while aiming
to provide things according to a service oriented paradigm. Altough the presented concept is
very innovative, the proposed solution could not be evaluated since that the the solution was
not implemented.\\

% ----------------------------------
% CLOUDTHINGS
% ----------------------------------
CloudThings \cite{zhou2013cloudthings} is an architecture that uses a common approach
to integrate Internet of Things and Cloud Computing. The proposed architecture is an online
platform which accommodates IaaS, Paas, and SaaS and allows system integrators and solution
providers to leverage a complete IoT application infrastructure for developing, operating and
composing IoT applications and services. The applications consists of three major modules,
the CloudThings service platform, that is a set of Cloud services (IaaS), allowing users to
run any applications on Cloud hardware. This service platform dramatically simplifies the
application development, eliminates need for infrastructure development and reduces management
and maintenance costs. The CloudThings Developer Suite is a set of Cloud service tools (PaaS) for
application development, such as Web service APIs, which provide complete development and
deployment capabilities to developers. The CloudThings Operating Portal is a set of Cloud
services (SaaS) that support deployment and handle or support specialized processing services.
To evaluate CloudThings, a smart home application based on a Cloud infrastructure was implemented.
In the application, the sensors read the home temperature and luminosity and the Cloud
stored and visualized the data, so that the user can view the smart home temperature and luminosity
anywhere. In particular, in these implementation the Cloud architecture was extended by inserting
a special layer for dynamic service composition. This middleware encapsulates sets of fundamental
services for executing the users service requests and performing service composition, such as process planning,
service discovery, process generation, process execution, and monitoring. This first implementation
also demonstrates that this middleware as a service releases the burden of costs and risks for users and
providers in using and managing those components.\\
% --------------------------------------------
% AUTOMATED DEPLOYMENT OF CLOUD APPLICATIONS
% --------------------------------------------
\subsection{Automated Deployment of Cloud Applications}
\label{sub:Automated Deployment of IoT Applications}
The effort put in the research to integrate the paradigms of Cloud Computing and Internet
of Things resulted in a essential contribution. The integration of these two paradigms allow
that IoT applications to have higher availability, scalability and to reduce the costs associated
with the maintenance of the application. But there are several issues regarding to the integration
between Cloud Computing and Internet of Things that must be addressed. In particular, due of the
heterogeneity of the IoT applications environments, it is hard for solution providers to efficiently
deploy and configure applications for a large number of users. Thus, automation of the management
tasks required by IoT applications is a key issue to be explored.\\

% ---------------------------------------------
% TOSCA
% ---------------------------------------------
TOSCA (Topology and Orchestration Specification for Cloud Applications) \cite{li2013towards} is
proposed in order to improve the reusability of service management processes and automate IoT application
ratified by OASIS in \cite{} deployment in heterogeneous environments. TOSCA is a new cloud standard to formally
describe the internal topology of application components and the deployment process of Cloud applications.
The structure and management of IT services is specified by a meta-model, which consists of
a \textit{Topology Template}, that is responsible for describing the structure of a service, then there
are the \textit{Artifacts}, that describe the files, scripts and software components necessary to be
deployed in order to run the application, and finally the \textit{Plans}, that defined the management process
of creating, deploying and terminating a service. The correct topology and management procedure can be inferred
by a TOSCA environment just by interpreting the topology template, this is known as ``declarative" approach.
Plans realize an ``imperative" approach that explicitly specifies how each management process should be done.
The topology templates, plans and artifacts of an application are packaged in a Cloud Service Archive (.csar file)
and deployed in a TOSCA environment, which is able to interpret the models and perform the specified management
operations. These .csar files are portable across different Cloud providers, which is a great benefit in terms
of deployment flexibility. To evaluate its feasibility, TOSCA was used in the specification of a typical
application in building automation. an application to control an Air Handling Unit (AHU). The common IoT
components, such as gateways and drivers will be modeled, and the gateway-specific artifacts that are
necessary for application deployment will also be specified. By archiving the previous specifications
and corresponding artifacts into a .csar file, and deploying it in a TOSCA environment, the deployment
of AHU application onto various gateways can be automated. As a newly established standard to counter
growing complexity and isolation in cloud applications environments, TOSCA is gaining momentum in industrial
adoption as well for academic interest.\\

% ------------------------------------------------------------------------------------
% COMBINING DECLARATIVE AND IMPERATIVE CLOUD APPLICATION PROVISIONING BASED ON TOSCA
% ------------------------------------------------------------------------------------
Breitenb\"{u}cher et al. \cite{breitenbucher2014combining} proposed to combine the two flavors of management
supported by TOSCA, \textit{declarative processing} and \textit{imperative processing}, in order to create
provisioning plans based on TOSCA topology models. The combination of both flavors would enable applications
developers to benefit from automatically provisioning logic based on declarative processing and individual
customization opportunities provided by adapting imperative plans. These provisioning plans are workflows that
can be executed fully automatically and may be customized by application developers after initial generation.
The approach enables benefits from both flavors that may leads to economical advantages when developing applications
with TOSCA. The motivating scenario that is used to evaluate this approach consists in a LAMP-based\footnote{LAMP
is a software stack that originally consists of the Linux system, the Apache HTTP Server, the MySql relational
database and the PHP programming language.} TOSCA application to be provisioned. The application implements
a Web-shop in PHP that uses a MySQL database to store product and customer data. The application consists
of two application stacks, one provides the infrastructure for the application logic and the other hosts
the database. Both ran on Amazon's public IaaS offering Amazon EC2. To measure the performance of the
deployment using the two-flavor approach, the time spent to generate provisioning plans regarding the
number of templates required by the application was measured. The results indicate that the required time
increases linearly with the number of templates.\\

Recently a growing number of organizations are developing Orchestrators, Design Tools and Cloud Managers
based on TOSCA. Juju is an Open Source TOSCA Orchestrator that can deploy workloads across public, private clouds,
and in physical machines. HP Cloud Service Automation is cloud management solution that supports declaratives
services design that are aligned with TOSCA modeling principles. GigaSpaces Cloudify orchestrates TOSCA Service
Templates using workflows to automate deployments and other DevOps automation processes. IBM Cloud Orchestrator
provides integrated tooling to create TOSCA applications, deploy them with custom polices, monitoring and scale
them in cloud deployments.\\
% ------------------------------------------------
%  SERVICE LEVEL AGREEMENTS IN CLOUD APPLICATIONS
% ------------------------------------------------
\subsection{Service Level Agreements in Cloud Applications}
\label{sub:subsection label}
Initially the objective of this work was focused in improve the deployment operation. However, during
the research of this work it became clear that the automation of the deployment of IoT applications
in smart places was an area that has a notorious progress in the developed work, standing out the development
of Cloud Orchestration tools based on TOSCA. Therefore, it was realized that the focus of
this work could be redirected to explore a field that still needs large research effort and is fundamental
for the IoT success \cite{atzori2010internet}, \textit{QoS} support. In order to guarantee an acceptable \textit{QoS}
for Cloud-based IoT applications, SLAs must be negotiated between service customers and services
providers in order to guarantee an acceptable \textit{QoS} for the service. Recently, \textit{QoS} support has
been a field where researchers are proposing new approches regarding policy-aware IoT applications based on TOSCA.\\

Waizenegger et al. \cite{waizenegger2013policy4tosca} proposed a mechanism to demonstrate how non-functional
requirements are defined in TOSCA using policies. TOSCA allows the specification of non-functional requirements
like cost, security, and environmental issues. However, TOSCA lacks detailed description of how to apply, design,
and implement policies. Waizenegger et al. propose two mechanisms for processing policies during the deployment and
management of Cloud services in TOSCA. The policies are described according a taxonomy where a Cloud service
policy is a tuple of elements that describe its behavior and effect. To perform the implementation of this
policy-specific logic, two approches were proposed: an Implementation Artifact Based Policy Enforcement (IA-Approach)
and a Plan-Based Policy Enforcement (P-Approach) \cite{waizenegger2013policy4tosca}. In the IA-Approach the existing
Implementation Artifacts - responsible to perform service management operations - are extended to support
policy enforcement implementations, thus these artifacts are comprised of two alternative implementations
for each service management operation: one that is policy enforcing and one that is not. In the P-Approach the
policies are enforced in an imperative way, for each policy the plan execute the appropriate steps to enforcing
the policy. Although both of these approaches allow the enforcement of policies, the approach used for realizing
this should be chosen with caution. The IA-Approach is more suitable for less complex scenarios and the
policies implemented with this approach can be reused very easily. On the other hand, the enforcement of
policies in more complex scenarios should be performed according the P-Approach, that allows to have a global
control of the triggered enforcement actions. The proposed solution could not be evaluated since that no
implementation was performed.\\

Breitenb\"{u}cher et al. \cite{breitenbucher2014policy} proposed an approach that enables automated provisioning
and management of composite Cloud applications in compliance with non-functional security requirements defined by
policies. In this approach the Management Planlet Framework \cite{breitenbucher2013policy} was extended to support
policy-aware provisioning and management based in Management Policies that bind non-functional security requirements
to the management tasks that enforce them. This paper also introduces the concept of Management Annotation
Policies, which defines the semantics and how the policy must be processed. The automation of provisioning and
management tasks are performed by Policy-aware Management Planlets, that execute these tasks considering the
Annotation Policies attached to each operation. The Framework also allows the Cloud providers and application
developers to specify their own policy-aware management logic in a flexible and reusable manner independently
from individual applications. The management logic can be defined in two ways, by applying Automated Management
Patterns or by manually creating Desired Application State Models. The Framework was evaluated in terms of
feasibility, performance, economics, limitations and extensibility. A prototype was successfully implemented based
to validate the concept technically based on a previous implementation of the Management Planlet Framework.\\
