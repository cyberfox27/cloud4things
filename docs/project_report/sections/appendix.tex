\section{Appendix}
\label{sec:appendix}
% ---------------------------------
%  WORK SCHEDULE
% ---------------------------------
\subsection{Work Schedule}
\label{sub:work_schedule}
In this section we propose a schedule that estimates the required time in future work that will be realized.
% schedule table
\begin{table}[h]
  \begin{tabular}{|c|c|c|}
    \hline
    \textbf{Milestone}                                                         & \textbf{Dates}          & \textbf{Total Days} \\ \hline
    Create the model of the solution's structure                                & 10/02/2015 - 17/02/2015  & 7          \\ \hline
    Implement a first release of the solution that delivers a useful feature.   & 18/02/2015 - 04/03/2015  & 14         \\ \hline
    Perform the tests and the evaluation.                                       & 04/03/2015 - 06/03/2015  & 2          \\ \hline
    Define what will be implemented in the next releases.                       & 07/03/2015 - 09/03/2015  & 2          \\ \hline
    Implement the next releases.                                                & 10/03/2015 - 11/04/2015  & 30         \\ \hline
    Perform the tests and the evaluation.                                       & 12/04/2015 - 14/05/2015  & 2          \\ \hline
    Write the dissertation document                                             & 15/05/2015 - 15/06/2015  & 30         \\ \hline
    Deliver the final version of the dissertation document                      & 16/06/2015               & 1          \\ \hline
  \end{tabular}
  \caption {Milestones and corresponding dates for the work of Master Thesis.}
\end{table}
% --------------------------------
%  RFID EVENT HANDLING EVALUATION
% --------------------------------
\subsection{RFID Event Handling Evaluation}
\label{sub:rfid_evaluation}
RFID has its particular characteristics, which means that traditional event processing systems
cannot support them. Furthermore, RFID events are temporal constrained \cite{wang2006bridging}.
Temporal constraints as the time interval between two events and the time interval for a single
event are critical to event detection. In order to assure the correctness of these received events,
some conditions must be defined regarding the time interval between the events. Temporal constraints
are not the only cause that can compromise the correctness of received the RFID events such as
collisions on the air interface, tag detuning and tag misalignement \cite{floerkemeier2004issues}.\\

The metric used to evaluate the RFID performance is proposed by Correia \cite{Correia:Thesis:2014}.
The \textit{PresenceRate} is a metric that takes in account the reported time of reading a RFID tag.
This metric measures the ratio between the time that a given object spent in a given area and the
expected spent time. Correia \cite{Correia:Thesis:2014} defined that the expected behaviour of this
metric is given by the following values:
\begin{itemize}
  \item \textit{PresenceRate = 0}: the object was not detected by the system;
  \item \textit{0 $<$ PresenceRate $<$ 1}: there are false negative readings and they were reported;
  \item \textit{PresenceRate = 1}: ideal scenario where the reported time is exactly the same as expected;
  \item \textit{PresenceRate $>$ 1}: there are false positive readings over-estimating the time
  spent by the object in the read area;
\end{itemize}

In order to perform a general evaluation of the system, we will establish a relation
between the \textit{PresenceRate} and the metrics of Cloud performance. The scenario where
the \textit{PresenceRate} is larger then 1 can be used to determine if the system is overloaded.
In this scenario the metrics of \textit{PresenceRate} and \textit{SystemLoad} must be compared to
verify this condition. If the \textit{SystemLoad} value is equal to 1, it means that the system
is operating in it maximum capacity and the amount of resources available is not enough to
process all the requests. Otherwise this comparison can determine if the system is overpowered.
If the \textit{PresenceRate} is equal to 1 and the \textit{SystemLoad} is close to 0, it means that the
available resources are more than the required to process all the requests.
