% -------------------------------------------------------------------------------------------------
% RELATED WORK
% -------------------------------------------------------------------------------------------------
\section{Related Work}
\label{sec:related_work}
Computing infrastructure in location for IoT applications is cost ineffective and presents low scalability.
Recently, the cloud paradigm allowed to move the computing infrastructure to the cloud providers,
making possible to increase the application scalability and also to reduce in a significant way the cost
related with the smart place infrastructure.

% ----------------------------------------
% INTERNET OF THINGS AND CLOUD COMPUTING
% ----------------------------------------
In RFID-based IoT applications, Guinard et al. \cite{guinard2011cloud} point out that the
deployment of RFID applications are cost-intensive mostly because they involve the
deployment of often rather large and heterogeneous distributed systems. As a consequence,
these systems are often only suitable for big corporations and large implementations and
do not fit the limited resources of small to mid-size businesses and small scale applications
both in terms of required skill-set and costs. To address this problem, Guinard et al. propose
a cloud-based solution that integrates virtualization technologies and the architecture of
the Web and its services. The case of study presented in the paper consists of an IoT application
that uses RFID technology to substitute existing Electronic Article Surveillance (EAS) technology,
such as those used in clothing stores to track the products. In this scenario they applied the
Utility Computing blueprint to the software stack - Fosstrak - required by the application using
the Amazon Web Services platform and the Amazon EC2 service. To evaluate the Cloud-based solution,
two prototypes was successfully implemented to prove that the pain points of the RFID applications
can be relaxed by adopting the proposed solution.

However, provisioning applications for Internet of Things still is an issue, because to the Virtual Machines
need to be manually configured and the deployment operation of those applications is specific for each
cloud provider.

% ---------------------------------------------
% AWS
% ---------------------------------------------
Amazon Web Services\footnote{http://aws.amazon.com/} (AWS) offers a variety of services to automate
the provisioning of the IT infrastructure at the Amazon Elastic Computing Cloud (EC2). VM Import/Export is a
service provided by AWS that enables to import virtual machine images from the development environment
to EC2 instances and export them back to the on-premisies development environment. This offering allows
to leverage the existing investments in the virtual machines that was built to meet the IT security,
configuration management, and compliance requirements by bringing those virtual machines into EC2 as
ready-to-use instances. The instances also can be exported to the on-premises virtualization infrastructure,
allowing to deploy workloads across the IT infrastructure. Another service provided by AWS is Elastic Beanstalk
that allows to quickly deploy and manage an application in the AWS cloud. The application is uploaded to AWS,
and Elastic Beanstalk automatically handles the details of capacity provisioning, load balancing,
scaling and application health monitoring. Elastic Beanstalk supports several types of applications,
including Java, Python, Ruby on Rails and Docker containers.

% ---------------------------------------------
% TOSCA
% ---------------------------------------------
TOSCA (Topology and Orchestration Specification for Cloud Applications) \cite{li2013towards} is a new
cloud standard to formally describe the internal topology of application components and the deployment
process of cloud applications. TOSCA is proposed in order to improve the reusability of service management
processes and automate IoT application ratified by OASIS in deployment in heterogeneous environments.
The structure and management of IT services is specified by a meta-model, which consists of
a \textit{Topology Template}, that is responsible for describing the structure of a service, then there
are the \textit{Artifacts}, that describe the files, scripts and software components necessary to be
deployed in order to run the application, and finally the \textit{Plans}, that defined the management process
of creating, deploying and terminating a service. The correct topology and management procedure can be inferred
by a TOSCA environment just by interpreting the topology template, this is known as ``declarative" approach.
Plans realize an ``imperative" approach that explicitly specifies how each management process should be done.
The topology templates, plans and artifacts of an application are packaged in a Cloud Service Archive (.csar file)
and deployed in a TOSCA environment, which is able to interpret the models and perform the specified management
operations. These .csar files are portable across different Cloud providers, which is a great benefit in terms
of deployment flexibility. To evaluate its feasibility, TOSCA was used in the specification of a typical
application in building automation. an application to control an Air Handling Unit (AHU). The common IoT
components, such as gateways and drivers will be modeled, and the gateway-specific artifacts that are
necessary for application deployment will also be specified. By archiving the previous specifications
and corresponding artifacts into a .csar file, and deploying it in a TOSCA environment, the deployment
of AHU application onto various gateways can be automated. As a newly established standard to counter
growing complexity and isolation in cloud applications environments, TOSCA is gaining momentum in industrial
adoption as well as academic interest.
