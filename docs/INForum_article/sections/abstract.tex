% -------------------------------------------------------------------------------------------------
% ABSTRACT
% -------------------------------------------------------------------------------------------------
\begin{abstract}
Smart places are an ecosystem composed of sensors - e.g. RFID - actuators - e.g. automatic doors -
and computing infrastructure - e.g. cloud servers - that are able to acquire data about the surrounding
environment and use that data to improve the experience of the people using the place.The data acquired
by the sensors needs to be collected, interpreted and transformed into information that is used to gather
knowledge about the smart place. A complete example of a platform that allows the transformation of sensor
data into information is Fosstrak, an open source RFID software that implements the EPC Network standards.
In this paper we propose Cloud4Things, a solution that automates the provisioning of RFID software in the
cloud by relying on configuration management tools that leverage existing stacks. We will perform a qualitative
evaluation of our solution based on Docker containers with other solutions, such as full Virtual Machines,
or tools implementing the TOSCA standards. The current prototype is able to support initial provisioning and
day-to-day operations stages in the lifecycle of a smart place.
\end{abstract}

\begin{keywords}
Smart places,
Automatic provisioning,
Container-based virtualization,
Configuration management tools,
Cloud applications
\end{keywords}
