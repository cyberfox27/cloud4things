\documentclass{../llncs2e/llncs}
\usepackage{inputenc}
\usepackage{graphicx}
\usepackage{caption}
% page numbering
\pagestyle{plain}
% change the spacing between the caption and the table
\captionsetup[table]{skip=10pt}
% -------------------------------------------------------------------------------------------------
% CLOUD4THINGS: automatic provisioning of smart places infrastructure
% -------------------------------------------------------------------------------------------------
\title{Cloud4Things: automatic provisioning of smart places infrastructure}
% -------------------------------------------------------------------------------------------------
% AUTHORS
% -------------------------------------------------------------------------------------------------
 \author{\Large Marcus Vin\'icius Paulino Gomes}
% -------------------------------------------------------------------------------------------------
% INSTITUTION
% -------------------------------------------------------------------------------------------------∫
 \institute{\large T\'ecnico Lisboa, Universidade T\'ecnica de Lisboa\\
 \email{\{marcus.paulino.gomes\}@tecnico.ulisboa.pt}}
% -------------------------------------------------------------------------------------------------
% DOCUMENT
% -------------------------------------------------------------------------------------------------
\begin{document}
\maketitle
% -------------------------------------------------------------------------------------------------
% ABSTRACT
% -------------------------------------------------------------------------------------------------
\begin{abstract}
Smart places are an ecosystem composed by sensors and actuators that are able to acquire knowledge
about its environment and also to adapt itself in order to improve the experience of the peoples
that lives in this environment. In this paper we propose Cloud4Things, a solution that automates
the provisioning of smart places infrastructure in the Cloud by relying on configuration management
tools. In our solution we also want that an efficient and portable infrastructure software stack,
in that way Cloud4Things will use Docker containers to virtualize the infrastructure stack. To
evaluate our solution we are using Fosstrak, an open source RFID software platform that implements
most of the Electronic Product Code standards. We will perform a qualitative evaluation by comparing
the required software stack by our solution to provisioning the infrastructure for an application
based on Fosstrak with other approaches, such as full Virtual Machines and TOSCA.
\end{abstract}
% -------------------------------------------------------------------------------------------------
% INTRODUCTION
% -------------------------------------------------------------------------------------------------
\section{Introduction}
\label{sec:introduction}
% -------------------------------------------------------------------------------------------------
% RELATED WORK
% -------------------------------------------------------------------------------------------------
\section{Related Work}
\label{sec:related_work}
% -------------------------------------------------------------------------------------------------
% SOLUTION ARCHITECTURE
% -------------------------------------------------------------------------------------------------
\section{Solution}
\label{sec:solution}
% -------------------------------------------------------------------------------------------------
% QUALITATIVE EVALUATION
% -------------------------------------------------------------------------------------------------
\section{Qualitative Evaluation}
\label{sec:qualitative_evaluation}
% -------------------------------------------------------------------------------------------------
% CONCLUSION
% -------------------------------------------------------------------------------------------------
\section{Conclusion}
\label{sec:conclusion}
% -------------------------------------------------------------------------------------------------
% BIBLIOGRAPHY
% -------------------------------------------------------------------------------------------------
%\bibliographystyle{ieeetr}
%\bibliography{}
\end{document}
