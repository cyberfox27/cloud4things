%!TEX root = ../dissertation.tex

\chapter{Introduction}
\label{chapter:introduction}
In recent years, computing is becoming more ubiquitous in the physical world. This notion where
computational elements are embedded seamlessly in ordinary objects that are connected through a
continuous network was introduced many years ago \cite{weiser1991computer}. The progress
towards ubiquitous computing has been slower than expected, technology advances such as the mobile
Internet contributes to achieve this vision in which computational devices are able to communicate
between themselves from any part of the world \cite{gubbi2013internet}. In this vision, an ubiquitous
system is composed of physical items that are continuously connected to the virtual world and can act as
remotely and physical access points to Internet Services \cite{mattern2010internet}.\\

However, there are some challenges that must be addressed in order to make these \gls{ubicomp} systems
truly ubiquitous \cite{caceres2012ubicomp}. An important concern regards about ubiquitous data: \textit{Where it is located?},
\textit{Who can access it?} and \textit{How much time this data should persist?}. Also, ubiquitous systems
are constantly interacting with the surrounding environment, thus these systems need to understand
the context in that they are inserted and also to adapt to the changes that occur in this environment.
Another import concern regards about the infrastructure burden of the ubiquitous systems. These
systems requires low-latency interaction with users and environments, which implies that at least part
of an \gls{ubicomp} application needs to be tightly bounded to the local infrastructure of the interacting
environment. This requirement for local infrastructure is a barrier in the adoption of ubiquitous
systems in a large-scale perspective.\\

This ubiquitous world is close to becoming reality thanks to the Utility Computing in the cloud
and the \gls{IoT}. In one hand, the utility computing provides the illusion of infinite computing
resources available on demand to the public users \cite{armbrust2010view}, which helps to reduce the
infrastructure burden of the ubiquitous systems. In the other hand the \gls{IoT} aims to solve a key
problem in wider adoption of ubiquitous systems, the tight coupling with a particular embedded
infrastructure. With the \gls{IoT} a variety of \textit{objects} or \textit{things} - such as \gls{RFID},
tags, sensors, actuators, etc. - will be able to interact with each other and cooperate with the
surrounding \textit{things} to reach common goals \cite{atzori2010internet}.\\

% Motivation
\section{Motivation}
\label{section:motivation}
This recent progress in the utility computing and the Internet of Things has been contributing to grow the
\gls{ubicomp} infrastructure. However, it is possible to identify new challenges \cite{caceres2012ubicomp}
for the construction of ubiquitous systems that arises from the integration of the Internet of Things
in the utility computing:

% Challenges
\begin{itemize}
  % Low-latency Interaction
  \item \textbf{Low-latency Interaction} is a key requirement of \gls{IoT} systems. In cloud-based
  solutions, part of the system's infrastructure is moved to the cloud - for instance the middleware layer,
  which is responsible for processing the information and take automatic decisions based on the results
  - is not possible to guarantee that the low-latency requirements of \gls{IoT} systems will be
  met for a cloud-based solution.
  % Scalable Data Storage
  \item \textbf{Data} is continuously generated by the \textit{things} that composes the \gls{IoT}
  system. That data must be stored, processed and presented in an seamless and efficient way. Several
  middleware solutions for \gls{IoT} rely on consistent transactions supported by \glspl{RDBMS}
  \cite{floerkemeier2007rfid}\cite{eisenhauer2010hydra}\cite{de2008socrades}, which unfortunately is
  not scalable \cite{hofmann2010cloud}. Since there is no industrial-grade solution for applications
  that rely on consistent transactions to write in multiple nodes at the same time, running high-volume,
  mission-critical transactional systems in the cloud is not a viable option.
  % Infrastructure Provisioning
  \item \textbf{Infrastructure Provisioning} usually was performed in a physical, isolated and vertical
  manner, in which hardware, networks, middleware and application logics are tightly coupled. Recently,
  \gls{IoT} has been adopted more and more in business and tends to weave into our daily life through the smart cities
  \cite{caragliu2011smart}\cite{schaffers2011smart}, this provisioning model presents some limitations
  that makes the delivery of new services an inefficient process, since that each time the same process
  has to be repeated to develop and deploy a new vertical solution.
  % Management
  \item \textbf{Management} of IoT systems is an issue that must be carefully addressed. In one hand
  is already proved that end-users are unable to manage their own personal computers systems well
  \cite{doll1988measurement}. Since there is no reason to believe that they will had a better
  performance at managing a much more complex system - such as an \gls{IoT} system - is reasonable to
  assume service providers will perform the management of the services . In the other hand, if we
  consider applications that are deployed in large-scale - such as smart cities - these managed
  services introduces new questions that must be answered. For instance, who will pay for this
  services and who will control this services?
  % Investment
  \item \textbf{Investment} in infrastructure is a relevant part of building an \gls{IoT} system,
  is this system deployed in a an small business or in a entire city. Is reasonable to assume that
  these investments need to be repaid. For instance, for businesses the return of investment is
  achieved with the improvement of its business processes, which can help to reduce the manufacturing
  cost of goods, to increase the efficiency of the supply chain, etc. However, for a large-scale
  system that is available for the general public a new model that is beneficial for both users and
  providers need to be established.
\end{itemize}

We believe that finding a solution to these problems is essential to the adoption of \gls{IoT} systems
in a large-scale perspective. However, some of these problems has a socio-economical nature - such as
the management and investment of \gls{IoT} systems - and will take time to resolve this issues.

% Related Work
\section{Related Work}
\label{section:related_work}
Recently a lot of research and effort has been dedicated to solve these existing problems. In this
section we will present a summary of the most relevant related work that address to solve the
problems of converging the Internet of Things and the utility computing:

% Infrastructure Provisioning
\subsection{Infrastructure Provisioning}
\label{sub:provisioning}
The research area for infrastructure provisioning of IoT solutions is the one that presents the most
notable progress until now.

% Soldatos
\subparagraph{Soldatos} et al. \cite{soldatos2012convergence} presented the idea of converging the IoT and the utility computing in the
cloud. The proposed architecture is the core concept of the OpenIoT Project\footnote{http://openiot.eu},
and is based on CoAP \cite{shelby2014constrained} and linked data. The cloud is used at infrastructure
level, which allows to measure the utility of the services provided by inter-connected objects.
% Distefano
\subparagraph{Distefano} et al. \cite{distefano2012enabling} proposes an conceptual architecture by mapping various
elements in both clouds and IoT to the three layers of cloud architecture (\gls{PaaS}, \gls{SaaS} and \gls{IaaS}).
In this proposal IoT resources are provided voluntarily by their owners, while management functions
- such as node management and policy enforcement - are viewed as peer functions of cloud infrastructure
management. IoT resources and cloud infrastructure are mashed up for applications, which are delivered
through \gls{SaaS}.
% CloudThings
\subparagraph{CloudThings} \cite{zhou2013cloudthings} is an architecture that uses a common
approach to integrate Internet of Things and Cloud Computing. The proposed architecture is an online
platform which accommodates \gls{IaaS}, \gls{PaaS}, \gls{SaaS} and allows system integrators and
solution providers to leverage the complete application infrastructure for developing, operating
and composing applications and services.
% IoT PaaS
\subparagraph{Li} et. al \cite{li2013efficient} proposed IoT PaaS, a cloud platform that supports
scalable IoT service delivery. Solution providers are able to deliver new solutions by leveraging
computing resources and platform services - domain mediation, application context management, etc.
- on the cloud. The proposed architecture aims to enable virtual vertical service delivery, for that
it has a multi-tenant nature which is designed to help at the isolation of the environments of
different solutions.\\
% Scalable Data Storage
\subsection{Scalable Data Storage}
\label{sub:data_storage}
Talk about the relevant work until now and mention the proprietary solutions for the cloud providers.
% Low-latency Interaction
\subsection{Low-latency Interaction}
\label{sub:low_latency_interaction}
Talk about this requirement through the different applications domain, i.e, each domain have its own
requirements for low-latency interaction.
% Objectives
\section{Objectives}
\label{section:objectives}
In this section I want to specify the application domain - RFID smart places - and describe the
objectives of our work.
% Document Structure
\section{Thesis Outline}
\label{section:outline}
The remainder of this document is organized as follows:
% Document structure item list
\begin{itemize}
  \item \textbf{Chapter \ref{chapter:background} Background}
  \item \textbf{Chapter \ref{chapter:implementation} Implementation}
  \item \textbf{Chapter \ref{chapter:methodology} Evaluation Methodology}
  \item \textbf{Chapter \ref{chapter:results} Evaluation Results}
  \item \textbf{Chapter \ref{chapter:conclusion} Conclusion}
\end{itemize}
