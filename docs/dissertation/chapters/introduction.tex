%!TEX root = ../dissertation.tex

\chapter{Introduction}
\label{chapter:introduction}
In recent years, computing is becoming more ubiquitous in the physical world. This notion where
computational elements are embedded seamlessly in ordinary objects that are connected through a
continuously available network was introduced many years ago by Mark Weiser \cite{weiser1991computer},
coining the term \gls{ubicomp}. Technology advances such as the mobile Internet contributes to
achieve this vision \cite{gubbi2013internet}. The \gls{IoT}, an ubiquitous system composed of
physical items that are continuously connected to the virtual world and can act as remotely and
physical access points to Internet Services \cite{mattern2010internet}. However, there are some
challenges that must be addressed in order to make these \gls{ubicomp} systems truly ubiquitous
\cite{caceres2012ubicomp}. An important concern regards about ubiquitous data: \textit{Where it is located?},
\textit{Who can access it?} and \textit{How much time this data should persist?}. Also, ubiquitous systems
are constantly interacting with the surrounding environment. Thus, these systems need to understand
the context in that they are inserted and also to adapt to the changes that occur in this environment.
Another import concern regards about the infrastructure burden of the ubiquitous systems. These
systems requires low-latency interaction with users and environments, which implies that at least part
of an \gls{ubicomp} application needs to be tightly bounded to the local infrastructure of the interacting
environment. This requirement for local infrastructure can be a barrier in the adoption of ubiquitous
systems in a large-scale perspective.\\

In one hand, the Utility Computing in the cloud provides the illusion of infinite computing resources
available on demand to the public users \cite{armbrust2010view}. This paradigm can help to reduce the
infrastructure burden of ubiquitous systems, while providing important features such as high availability
and high scalability.\\

In the other hand, the \gls{IoT} aims to solve a key problem in wider adoption of ubiquitous systems,
the tight coupling with a particular embedded infrastructure. With the \gls{IoT} a variety of
\textit{objects} or \textit{things} - such as \gls{RFID} tags, sensors, actuators, etc. - will be
able to interact with each other and cooperate with the surrounding \textit{things} to reach common
goals \cite{atzori2010internet}.\\

% Applications
\section{Applications}
\label{sec:applications}
The Internet of Things offers a great potential that makes possible the development of a huge number
of applications. There are several environments and domains where \gls{IoT} applications are expected
to improve the quality of life of the people that lives and works in these environments and also
will provide competitive advantage against current solutions. Currently, the application domains that
promises to play a big role in the adoption of \gls{IoT} are:

% Smart Environments
\subparagraph{Ambient Intelligence} are environments that uses the intelligence of the objects within
itself to become a more comfortable and efficient environment. For instance, a room that has its
temperature adapted according the weather, an office that has its lights adapted according the time
of the day and an automated industrial plant where its is possible to monitoring the production
progress and to see the  possible side-effects of a production delay due to a malfunction in a device.
% Logistics
\subparagraph{Logistics} applications are able to perform real-time information processing based
on technologies such as \gls{RFID} and \gls{NFC}. This information allows to achieve real-time monitoring
of almost every phase of the supply chain, ranging from raw material purchasing, transportation,
storage, distribution and after-sales services.
% Transportation
\subparagraph{Transportation} covers a wide area, from personal vehicles to trains, mobile ticketing
and transportation of goods. Cars, trains and buses are equipped with sensors, actuators and computational
power that are able to provide information about the status of the vehicle, improve the navigation and
even to perform collision avoidance. Regarding the transportation of goods, it is possible to monitoring
the conservation status of perishable goods - temperature, humidity, etc. -  during its transportation.
In public transportation, NFC tags can be used to provide information about the available services
- costs, schedule, etc.
% Healthcare
\subparagraph{Healthcare} is a domain that can benefit in a significant way from the \gls{IoT}
technologies. For instance, \gls{RFID} tags can be used to monitoring the position of patients,
hospital staff and also to control the inventory of materials. Sensors can be used to monitoring
in real-time patient conditions, hospital environment conditions - temperature, air quality, etc.
Moreover, these technologies will provide the ability to perform patient identification, staff authentication,
access control and also to perform automatic data collection in order to improve the efficiency of
the operations in the healthcare facilities.\\

Beyond of the presented examples, the \gls{IoT} field covers many other domains and they will
continuing to growing. To give a more abstract view over those domains, we propose the term
\textit{smart places}, that can be defined as the unification of these several application fields.

% Motivation
\section{Motivation}
\label{section:motivation}
Challenges \cite{caceres2012ubicomp} for the construction of ubiquitous systems that combines from
the integration of the Internet of Things in the utility computing:

% Challenges
\begin{itemize}
  % Low-latency Interaction
  \item \textbf{Low-latency Interaction} is a key requirement of \gls{IoT} systems. This requires that
  both network access latency and data transmission latency be reduced. However, in cloud-based solutions,
  part of the system's infrastructure is moved to the cloud - for instance the middleware layer,
  which is responsible for processing the information and take automatic decisions based on the results.
  However, will a cloud-based approach able to meet the low-latency requirements of \gls{IoT} systems?
  % Scalable Data Storage
  \item \textbf{Data} is continuously generated by the \textit{things} that composes the \gls{IoT}
  system. These data must be stored, processed and presented in a seamless and efficient way.
  Several middleware solutions for \gls{IoT} \cite{floerkemeier2007rfid}\cite{eisenhauer2010hydra}\cite{de2008socrades}
  relies on database management systems that are very efficient, but will those systems able to
  handle with the huge volume of data generated by \gls{IoT} applications?
  % Service Delivery
  \item \textbf{Service Delivery} usually was performed in a physical, isolated and vertical
  manner, in which hardware, networks, middleware and application logics are tightly coupled. Recently,
  \gls{IoT} has been adopted more and more in business and tends to weave into our daily life, for instance
  through the smart cities \cite{caragliu2011smart}\cite{schaffers2011smart}. This provisioning model presents some
  limitations that makes the delivery of new services an inefficient process, since that each time the
  same process has to be repeated to develop and deploy a new vertical solution.
  % Management
  \item \textbf{Management} of IoT systems is an issue that must be carefully addressed. In one hand
  it is already proved that end-users are unable to manage their own personal computers systems well
  \cite{doll1988measurement}. Since there is no reason to believe that they will had a better
  performance at managing a much more complex system - such as an \gls{IoT} system - is reasonable to
  assume that service providers will perform the management of the services . In the other hand, if we
  consider applications that are deployed in large-scale - such as smart cities - these managed
  services introduces new questions that must be answered. For instance, who will pay for this
  services and who will control this services?
  % Investment
  \item \textbf{Investment} in infrastructure is a relevant part of building an \gls{IoT} system,
  is this system deployed in a small business or in an entire city. Is reasonable to assume that
  these investments need to be repaid. For instance, for businesses the return of investment is
  achieved with the improvement of its business processes, which can help to reduce the manufacturing
  cost of goods, to increase the efficiency of the supply chain, etc. However, for a large-scale
  system that is available for the general public a new model that is beneficial for both users and
  providers need to be established.
\end{itemize}

We believe that finding a solution for these problems is essential for the adoption of \gls{IoT} systems
in a large-scale perspective. However, some of these problems has a socio-economical nature - such as
management and investment of \gls{IoT} systems - and will take time to resolve this issues.

% Objectives
\section{Objectives}
\label{section:objectives}
The objective of this work is to determine if a cloud-based solution is able fulfill the fundamental
requirements of Internet of Things applications. In this work we will focus to determine if such approach
can meet the requirement of low-latency interaction and data storage performance. Since smart places
presents different requirements according its domain, our efforts will be focused in determine if a
cloud-based solution is suitable for a smart place that relies on the RFID technology \cite{want2006introduction}.\\

\subsection{Domain}
\label{sub:domain}
Our smart place is an automated warehouse where products - which are transported through automated
guided vehicles - are tagged with \gls{RFID} tags that can be identified by \gls{RFID} readers.
Through the data collected by these readers is possible to gather knowledge about the smart place.
For instance, is possible to determine which products enters or leaves the warehouse. A complete example
of a platform that allows the transformation of that data into information is Fosstrak\footnote{\url{http://fosstrak.github.io/}},
an open source \gls{RFID} software that implements the \gls{EPC} Network standards.\\

To accomplish our objectives, in this work we will follow two approaches and determine which of them
is more adequate to deploy a \gls{RFID} application. The first is a more traditional approach, where
all the application middleware is provisioned in the cloud. The second, is an approach based on the
Fog Computing platform \cite{bonomi2012fog}, which extends the cloud paradigm to the edge of the
network.\\

Based in the proposal of the Fog Computing platform, our initial hypothesis is that a fog-based approach
will present a best overall performance and as consequence will be more adequate to deploy the \gls{RFID}
application in the smart warehouse. 

% Document Structure
\section{Thesis Outline}
\label{section:outline}
The remainder of this document is organized as follows:
% Document structure item list
\begin{itemize}
  % Background
  \item \textbf{Chapter \ref{chapter:background}. Background} summarizes the relevant work in the field and
  introduces some key concepts that supports our work such as a description of the Fog Computing paradigm,
  the EPCGlobal Network and Fosstrak platform, etc.
  % Solution
  \item \textbf{Chapter \ref{chapter:solution}. Solution} presents the approaches adopted to deploy
  and provisioning the smart place software stack, i.e the software stack required to deploy
  an application based on the Fosstrak platform.
  % Implementation
  \item \textbf{Chapter \ref{chapter:implementation}. Implementation} details the implementation
  for the proposed solution, the deployment approaches, the provisioning strategies, the virtualization
  technologies, etc.
  % Evaluation
  \item \textbf{Chapter \ref{chapter:evaluation}. Evaluation} describes the experiments made to meet
  the defined objectives and presents an analysis of the obtained results.
  % Conclusion
  \item \textbf{Chapter \ref{chapter:conclusion}. Conclusion} summarizes the presented work,
  presents the main conclusion and some important research points for future work.
\end{itemize}
