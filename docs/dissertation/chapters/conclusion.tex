%!TEX root = ../dissertation.tex

\chapter{Conclusion}
\label{chapter:conclusion}
The present work explored the deployment of \gls{IoT} applications for smart warehouses based on the
\gls{RFID} technology with two different approaches to deploy: one based in a traditional cloud
deployment approach (cloud-based) and other according the Fog Computing platform (fog-based). More
specifically, the present work focuses to determine if a cloud-based approach is able to meet the
low-latency requirements of many \gls{IoT} applications, since that low-latency is an essential
requirement of \gls{IoT} applications. If a cloud-based approach is not able to meet the network latency
requirements for those applications, the cloud platform is not a viable option to perform the
provisioning of \gls{IoT} applications.\\

To improve the provisioning of \gls{RFID} middleware in the cloud, we developed a mechanism based on
Docker containers and the Chef tool that automates the installation and configuration of the modules
that composes the Fosstrak platform \gls{RFID} middleware. This mechanism was of extreme
importance, because it allowed us to perform the application provisioning of the cloud instances in
a very efficient way. Although our experiments were conducted in a single cloud provider, the developed
mechanism gave us the flexibility to choose between several cloud providers to provision the
\gls{RFID} middleware.\\

Regarding the system evaluation, we defined two methodologies for evaluate the latency of an event that
occurs in the physical space and the data storage performance for the Fosstrak platform. With the
methodologies proposed, we were able to compare the event latency performance for both cloud-based
and fog-based approaches. We defined two experiments to evaluate the latency performance of the
deployment approaches. The obtained results shows that the event latency performance presented better
results when the application was deployed according the fog-based approach. However, we identified
some issues regarding the behavior of a Fosstrak module (\gls{ALE}) that affected the performance
of the event latency for both deployment approaches. Regarding the data storage performance of the
RFID middleware, the results show that the Fosstrak platform is able to process with an acceptable
performance the amount of data that is generated in a smart warehouse.

% Contributions
\section{Contributions Summary}
\label{sec:contributions}

\subparagraph{RFID Smart Place Deployment.}
\label{subp:rfid_smart_place_architecture}
A deployment approach based on the Cloud Computing platform for EPCGlobal compliant \gls{RFID}
middleware platforms. We propose an architecture that focuses to improve the network latency
performance of \gls{RFID} applications by distributing the middleware components across the fog and
the cloud.
%
A mechanism that automates the provisioning of \gls{RFID} application middleware in the cloud.
The provisioning mechanism allows to provisioning the Fosstrak modules in several cloud
providers. Furthermore, it is based on Docker container images to provisioning the application
stack, it is not specific for the Fosstrak middleware and can be extended for other EPCGlobal
compliant \gls{RFID} platform.
%
To provisioning the Fosstrak software stack, we developed a set of Docker images that used to
create the Docker containers with the Fosstrak modules, namely \gls{ALE}, Capture Application,
\gls{EPCIS} Repository and MySQL database. The images are open-source and available at
Docker Hub.
%
Our provisioning mechanism was implemented through the Chef configuration management tool. Since the
resources for provisioning the stack using Chef does not exists, we defined a set of \textit{recipes}
and \textit{roles} that allow to deploy and configure the Fosstrak software stack.

\subparagraph{Interaction Latency Evaluation.}
\label{subp:event_latency_performance_eval}
Experiments were performed in order to find the best cloud-based deployment approach that meets the
low-latency requirements of \gls{RFID} applications. Moreover, we compared both cloud-based and fog-based
approaches based on the \textit{Event Cycle} metrics in order to determine how the deployment
approach affects the performance of the \textit{Event Cycle} stages.

% Future Work
\section{Future Work}
\label{sec:future_work}
In the present work, we achieved our initial goals and determined that Utility Computing is adequate
to deploy a smart place application based in \gls{RFID} technology, both in cloud and fog configurations.
However, our solution is not perfect and there some aspects that can be improved in the future.

% Fog Implementation
\subparagraph{Fog Implementation.}
\label{subp:fog_impl}
Our solution proposes that the \gls{RFID} application is deployed following a fog-based approach.
This means that we need to have a cloud close to the ground and this cloud must meet the same
requirements of a remote cloud such as high scalability, security and multi-tenancy. Unfortunately,
we were not able to implement a fog that meet these requirements and in our implementation the fog
was built on top of a traditional Virtual Machine. In the future, the fog needs to be correctly
implemented providing all the features of the remote cloud and in addition features such as
location-awareness, mobility support and geo-distribution.

% Containers Deployment
\subparagraph{Containers Deployment.}
\label{subp:containers_impl}
In the current implementation we used Docker containers to provisioning the Fosstrak software stack.
In the evaluation of our solution we deploy the containers in a \gls{EC2} \gls{VM}, which overlays two
different mechanisms of virtualization. Although we still are able to take advantage of some benefits
from the containers such as the portability, other benefits such as the low I/O and disk space are
hidden by the \gls{VM} hypervisor. A future improvement that can be made is to perform the deployment
of the containers on top of the bare-metal or in a cloud-based container service - e.g. Google Kubernetes
\footnote{\url{http://kubernetes.io/}} or \gls{AWS} \gls{EC2} Container Service\footnote{\url{https://aws.amazon.com/ecs/}} -
in order to improve the overall performance of the solution.

% Cloud Providers Evaluation
\subparagraph{Cloud Providers Evaluation.}
\label{subp:cloud_eval}
The evaluation was performed only in \gls{AWS} \gls{EC2} instances. For the future is important to
evaluate our solution in other cloud providers to compare which offers the best cost/performance
relation.

% Latency Interaction Evaluation
\subparagraph{Latency Interaction Evaluation.}
\label{subp:latency_eval}
To evaluate the latency interaction, we defined only two different \textit{ECspecs}. For the future work,
we want to evaluate the latency performance for our solution with \textit{ECspecs} that presents smaller
periods in order to determine which specification is more suitable for our solution.

% Latency Interaction Evaluation
\subparagraph{Evaluation Scenario.}
\label{subp:scenario_eval}
In the evaluation scenario we used a virtual \gls{RFID} reader instead of a physical one, which does
not allow reproducing the environment conditions of a real smart warehouse such as interferences in
the \gls{RFID} tags antennas, network bandwidth variations, etc. However, in the evaluation experiments
we used for some experiments traces from the work developed by Correia et. al \cite{Correia:Thesis:2014},
which have the real data traces mentioned above. A future improvement is to conduct the system
evaluation in a real scenario in order to have more accurate results.

% Multi Domain Evaluation
\subparagraph{Multi Domain Evaluation}
\label{subp:multi_domain_evaluation}
In the present work, we confirmed that the Utility computing platform is adequate to deploy a
warehouse application based in RFID technology that requires low latency interaction for both cloud
and fog configurations. But it will be this approach the best choice for all application domains?
Since that IoT covers several domains (as described in \ref{table:smart_places_characteristics}), in
the future we want to perform a multi domain evaluation in order to determine if the Utility
Computing is adequate to deploy the applications for these domains.

% Summary
\section{Cloud infrastructure for Smart Place applications}
\label{sec:conclusion_summary}
With this work we have contributed to the validation of the suitability of the Cloud for Things-based
applications. We believe that using the cloud infrastructure to support smart place applications will be the
most adopted approach, even for applications that have strict requirements such for low-latency
and context-awareness. The flexibility provided by the cloud paradigm will allow that
\gls{IoT} applications from several domains may be deployed in a cloud infrastructure and take
advantage of the benefits offered by this paradigm. However, reliable and fast network connections
are a precondition. Another way to improve would be to have utility computing principles near the
edge of the network, near the devices, as represented by the fog approach in this work.
