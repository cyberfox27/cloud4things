%!TEX root = ../dissertation.tex

\chapter{Implementation}
\label{chapter:implementation}

\section{Smart Place}
\label{sec:Smart Place}


\section{Provisioning}
\label{sec:provisioning}
The current implementation of Cloud4Things relies on the Chef tool. The recipes that describe our
infrastructure are based on cookbooks that are available on the Chef Supermarket4. These recipes
describe how our software stack - are provisioned in the cloud instances. In our current prototype,
we will use Docker containers to provisioning the smart place software and we chose to use Amazon Web
Services as cloud provider. To provisioning the resources in the Amazon EC2 instances we will use knife,
a command-line tool developed by Chef that provides an interface between a local Chef repository and
the Chef server. The provisioning workflow is illustrated in Figure 3. In a development environment
the Docker images are built and then uploaded to the Docker Registry repository (1). The provisioning
of the cloud resources is described in the cookbooks that are uploaded to the Chef server (2). The
provisioning request (3) is performed using knife - knife has a plugin for EC2 that allows to describe
the image type, the instance type and the policies that need to be applied on each provisioned node.
Then the Chef client runs the configuration recipes that are pulled from the Chef server (4). In our
solution these configuration recipes describe that our nodes must have a set of Docker containers
running on it. The Chef client pulls the Docker images from the remote repository, build the containers
based on those images and finally applies the configuration that is associated to each container.
