%!TEX root = ../dissertation.tex

\chapter{Background}
\label{chapter:background}

% Related Work
\section{Related Work}
\label{section:related_work}
Recently a lot of research and effort has been dedicated to solve these existing problems. In this
section we will present a summary of the most relevant work that address to solve the problems of
converging the Internet of Things and the utility computing:

% Infrastructure Provisioning
\subsection{Infrastructure Provisioning}
\label{sub:provisioning}
The research area for infrastructure provisioning of IoT solutions is the one that presents the most
notable progress until now.

% Soldatos
\subparagraph{Soldatos} et al. \cite{soldatos2012convergence} presented the idea of converging the IoT
and the utility computing in the cloud. The proposed architecture is the core concept of the OpenIoT
Project\footnote{http://openiot.eu}, and is based on CoAP \cite{shelby2014constrained} and linked data.
The cloud is used at infrastructure level, which allows to measure the utility of the services provided
by inter-connected objects.
% Distefano
\subparagraph{Distefano} et al. \cite{distefano2012enabling} proposed a conceptual architecture by mapping various
elements in both clouds and IoT to the three layers of cloud architecture (\gls{PaaS}, \gls{SaaS} and \gls{IaaS}).
In this proposal IoT resources are provided voluntarily by their owners, while management functions
- such as node management and policy enforcement - are viewed as peer functions of cloud infrastructure
management. IoT resources and cloud infrastructure are mashed up for applications, which are delivered
through \gls{SaaS}.
% CloudThings
\subparagraph{CloudThings} \cite{zhou2013cloudthings} is an architecture that uses a common
approach to integrate Internet of Things and Cloud Computing. The proposed architecture is an online
platform which accommodates \gls{IaaS}, \gls{PaaS}, \gls{SaaS} and allows system integrators and
solution providers to leverage the complete application infrastructure for developing, operating
and composing applications and services.
% IoT PaaS
\subparagraph{Li} et. al \cite{li2013efficient} proposed IoT PaaS, a cloud platform that supports
scalable IoT service delivery. Solution providers are able to deliver new solutions by leveraging
computing resources and platform services - domain mediation, application context management, etc.
- on the cloud. The proposed architecture aims to enable virtual vertical service delivery, for that
it has a multi-tenant nature which is designed to help at the isolation of the environments of
different solutions.

% Scalable Data Storage
\subsection{Scalable Data Storage}
\label{sub:data_storage}
Since that no general solution for scalable data storage and retrieval in the cloud was developed,
cloud providers started to implement its own solutions.

\subparagraph{Google Big Table, Facebook Cassandra and  Amazon Dynamo} \cite{chang2008bigtable} \cite{lakshman2010cassandra}
\cite{decandia2007dynamo} are key-value stores - \gls{NoSQL} databases - that has the ability to horizontally scale - i.e,
distribute both data and load of simple operations through many servers - but it has a weaker concurrency model
than the ACID transactions of most \glspl{RDBMS} systems \cite{cattell2011scalable}.

\subparagraph{\gls{PIQL}} \cite{armbrust2010piql} is a SQL-like API built to run on top of existing
performance predictable key-value stores, that provides many of the benefits of using a traditional
\gls{RDBMS}, such as the ability to express the queries in a declarative way, automatic data
parallelism, physical data independence and automatic index selection and maintenance, all while
maintaining the soft real-time guarantees on application performance that come from the underlying
key-value store.\\

Recently, some progress has been reached as regards scalable storage for \gls{RDBMS} systems. Although
most of the work still are in development, it is possible to highlight some solutions that are in a more
mature state.

\subparagraph{MySQL Cluster} \cite{ronstrom2004mysql} is an in-memory clustered distributed \gls{RDBMS}.
Compared with the MySQL implementation it works by replacing the InnoDB engine with the NDB - a proprietary
distributed layer from MySQL. MySQL Cluster is built on top of a shared-nothing architecture and includes
features such as failover, node recovery, synchronous data replication and no single point of failure.
MySQL Cluster seem to be the solution that scales to more nodes than other \gls{RDBMS} - 48 is the limit.
However, it was reported that after scaling up to a few dozen nodes it starts to running into
bottlenecks \cite{bunch2010evaluation}.

\subparagraph{VoltDB} \cite{stonebraker2013voltdb} is a \gls{RDBMS} designed for performance and scalability.
VoltDB assumes a multi-node cluster architecture where the tables are partitioned over multiple servers.
Tables can be replicated over servers - e.g. for fast access to data - shards are always replicated -
to recovery the data in case of a node crash - and database snapshots are supported. Currently some
features still are missing - online schema changes are limited and asynchronous \gls{WAN} replication and
recovery are not yet implemented - but in its current implementation VoltDB already presents some
features that improves the performance of SQL execution, as result the number of nodes that are needed
to support a given application load can be reduced in a significant way.

% Low-latency Interaction
\subsection{Low-latency Interaction}
\label{sub:low_latency_interaction}
Finding a general solution for the low-latency requirement of \gls{IoT} systems is not a easy task.
Since there are several application domains, is natural that each application domain requires
that different values for network latency and data transmission latency are guaranteed.
% Talk about fog computing

 % Concepts
 \section{Concepts}
 \label{sec:Concepts}

% EPC Network
\subsection{EPC Network}
\label{sub:epc_network}
For our work we will focus on software for an RFID smart place. The software to use will be based on
the GS1 EPCglobal standards for RFID.\\

The EPC Application Level Events (ALE) provides interfaces to filter and collect data. The EPC
Information Services (IS) provides interfaces to interpret, store persistently and query captured
EPC event data. The EPC Low-Level Reader Protocol (LLRP), that is not demonstrated in this
architecture, provides an interface to configure and control RFID readers.

% Fosstrak
\subsection{Fosstrak}
\label{sub:fosstrak}

% Containers
\subsection{Containers}
\label{sub:containers}
Docker5 is an open source project to pack, ship and run any application as a lightweight container.
Docker containers are hardware-agnostic and platform-agnostic, this means that these containers can
run anywhere, from a laptop to a EC2 compute instance. Since Docker is based in Linux Containers (LXC),
the virtualization is performed at operating-system level, di↵erent of hypervisor-based solutions where
the virtualization is performed at hardware- level. While the e↵ect of both types of virtualization
are similar, the virtualization at the operating-system level provides significant benefits compared
to hypervisor-based solutions. Docker containers are small, they have low memory and CPU overhead,
they also are portable between di↵erent virtualization environments.\\

% TODO: Move to implementation chapter
In our solution, Docker containers are used to provisioning the software stack of the Fosstrak platform.
A complete installation of Fosstrak requires a compatible Java SDK, a full MySQL database and a Apache
Tomcat server. In order to improve the application scalability we are provisioning a single container
for each component of the Fosstrak platform, the EPCIS repository, the Capture application, the ALE
server, and also for the MySQL database.\\

By default each container runs a process that is isolated from the other processes that are executed
in the same environment. In order to connect the different modules of the Fosstrak, our containers are
linked through the linking system6 provided by Docker. This mechanism creates a secure tunnel between the
containers, allowing the recipient container to access select data about the source container. For instance,
if we have a web server container linked to a database container, the web server container is able to access
information about the database container.\\

Another benefit that Docker platform provides is the Docker Registry service, a public repository
that stores Docker images used to create the containers. In our solution we built the Docker images
of the Fosstrak modules and published them in Docker registry to later be used to create our containers.

% Configuration Management Tools
\subsection{Configuration Management Tools}
\label{sub:cm_tools}
Chef is a configuration management tool that allows to describe the infrastructure as code.
In that way it is possible to automate how the infrastructure is built, deployed and managed.
Chef architecture is composed of the Chef Server - that stores the recipes and other configuration
data - and the Chef Client - that is installed in each server, VM or container, i.e, the nodes that
are managed with Chef. The Chef client periodically pulls Chef server latest policy and state of the
network, and if anything on the node is out of date, the client update its state in order to be
consistent with the latest policy.\\

Chef was built from the ground with the cloud infrastructure in mind. With Chef, is possible to
dynamically provision and de-provision the application infrastructure on demand to keep up with
peaks in usage and track. For instance, Chef offers several plugins for provisioning cloud resources
in different hosts such as Amazon EC2, Google Compute Engine and OpenStack.

% Fog Computing
\subsection{Fog Computing}
\label{sub:Fog Computing}
