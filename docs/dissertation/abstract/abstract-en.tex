%!TEX root = ../dissertation.tex

\begin{otherlanguage}{english}
\begin{abstract}
% Enable page numbering
\abstractEnglishPageNumber

Smart places are an ecosystem composed of sensors, actuators and computing infrastructure that acquires
data about the surrounding environment and use that data to improve the experience of the people
interacting with the place. The smart place runs an Internet of Things (IoT) application that transforms
raw sensor data into informed action. For instance, RFID readers can detect a tagged object approaching
and an automatic door is opened after the event is processed in a dedicated server.\\

Usually, IoT applications are latency-sensitive because actions need to be done in a timely manner and to
meet this requirement these applications are provisioned close to the physical place, which represents
an infrastructure burden because it is not always practical to deploy a physical server at location.
The Utility Computing in the cloud can help to solve this issue. However, the latency requirements
must be carefully assessed. The Fog Computing is a recent concept that brings the cloud close to the
ground, providing low latency communication for applications and services.\\

The present work implemented two deployment approaches for IoT applications based in the Cloud and in
the Fog concepts. Our scenario is an automated warehouse that uses the Fosstrak platform - an
open-source implementation of RFID event processing software - to track the objects in the place.
We compared the event latency performance of both approaches and also the data storage performance.\\

The results show that a fog-based approach is more adequate for latency-sensitive applications,
presenting a better performance when compared with a cloud-based approach.\\

% Keywords
\keywords{Internet of Things, Cloud computing, Fog computing, Application Deployment, RFID, Fosstrak Platform}
\end{abstract}
\end{otherlanguage}
