%!TEX root = ../dissertation.tex

\begin{otherlanguage}{english}
\begin{abstract}
% Enable page numbering
\abstractEnglishPageNumber

Smart places are an ecosystem composed of sensors - e.g. RFID - actuators - e.g. automatic doors - and
computing infrastructure - e.g. servers - that are able to acquire data about the surrounding
environment and use that data to improve the experience of the people interacting with the place. The
smart place runs an Internet of Things (IoT) application that is able to transform this data into knowledge.
Usually, IoT applications are latency sensitive and to meet this requirement these applications are provisioned
in the place, which represents an infrastructure burden for this type of applications.
The Utility Computing in the cloud can help to solve this problem and also provides other benefits
for IoT applications such as high scalability and high availability.\\

The present work implemented two deployment approaches for IoT applications that are based in the cloud:
a fog-based approach and a traditional cloud approach. Our base scenario is an automated warehouse that
uses the RFID technology - based on the Fosstrak platform - to track the objects that are moving
inside the warehouse. We compared the network performance for both approaches and also the data storage
performance of the RFID platform.\\

The results show that a fog-based approach is more adequate for latency sensitive applications, presenting
a better overall performance for the network latency when compared with the cloud approach. Regarding
the data storage performance, the results show that the Fosstrak platform is able to handle with the
data generated by a smart warehouse.\\

% Keywords
\keywords{Internet of Things, Fog computing, Cloud computing, Application Deployment, RFID, Fosstrak Platform}
\end{abstract}
\end{otherlanguage}
