%!TEX root = ../dissertation.tex

\begin{otherlanguage}{english}
\begin{abstract}
% Enable page numbering
\abstractEnglishPageNumber

Smart places are an ecosystem composed of sensors, actuators and computing infrastructure that acquires
data about the surrounding environment and use that data to improve the experience of the people
interacting with the place. The smart place runs an Internet of Things (IoT) application that transforms
raw sensor data into informed action. For instance, RFID readers can detect a tagged object approaching
and an automatic door is opened after the event is processed in a dedicated server.\\

Usually, IoT applications are latency-sensitive because actions need to be done in a timely manner. To
meet this requirement these applications are usually provisioned close to the physical place, which represents
an infrastructure burden because it is not always practical to deploy a physical server at a location.
Utility Computing in the Cloud can solve this issue. However, the latency requirements must be carefully assessed.
Fog Computing is a recent concept that brings the cloud close to the ``ground'' (i.e close to devices at the edge of the network),
aiming to provide low latency communication for applications and services.\\

The present work implemented an automatic provisioning mechanism to deploy IoT applications according
in an Utility Computing platform. Our demonstration scenario is an automated warehouse that uses a
RFID event processing software to track objects in the facilities. We compared the event latency
performance of both approaches and data storage performance.\\

The results confirm that a fog-based approach is more adequate for latency-sensitive applications,
presenting a better performance when compared with a cloud-based approach.\\

% Keywords
\keywords{Internet of Things, Cloud computing, Fog computing, Application Deployment, RFID, Fosstrak Platform}
\end{abstract}
\end{otherlanguage}
