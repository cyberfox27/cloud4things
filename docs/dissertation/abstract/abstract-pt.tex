%!TEX root = ../dissertation.tex

\begin{otherlanguage}{portuguese}
\begin{abstract}
% Enable page numbering
\abstractPortuguesePageNumber

Espa\c{c}os inteligentes s\~ao um ecosistema composto de sensores - e.g. RFID - actuadores - e.g.
portas autom\'aticas - e.g. infra-estrutura computacional - e.g. servidores - que adiquirem dados
do ambiente em que est\~ao inseridos e usam estes dados para melhorar a experi\^encia das pessoas
que interagem com este lugar. O espa\c{c}o inteligente possui uma aplica\c{c}\~ao para a Internet das
Coisas (IoT) que \'e capaz de transformar estes dados em conhecimento. Geralmente, estas aplica\c{c}\~ao
s\~ao \textit{latency sensitive} e de modo a cumprir estes requisitos, a infra-estrutura necess\'aria
para aprovisionar a aplica\c{c}\~ao tem de ser local, o que requere um alto investimento.
A \textit{Utility Computing} na nuvem pode ajudar a resolver este problema, al\'em fornecer
benef\'icios como alta escalabilidade e alta disponibilidade.\\

Neste trabalho foram implementadas duas abordagens de \textit{deployment} baseadas na cloud para
aplica\c{c}\~oes IoT: uma abordagem baseda no computa\c{c}\~ao em nevoeiro e outra na abordagem
tradicional da nuvem. Nosso cen\'ario \'e um armaz\'em automatizado que utiliza a tecnologia
RFID - baseada na plataforma Fosstrak - para rastrear os objectos que se encontram no armaz\'em.
N\'os comparamos a performance da rede para ambas abordagens e tamb\'em a performance do mecanismo
de armazenamento de dados da plataforma Fosstrak.\\

Os resultados obtidos mostraram que a aboradagem baseada em nevoeiro \'e mais adequada para
aplica\c{c}\~oes \textit{latency sensitive}, apresentando uma melhor performance para a lat\^encia
da rede quando comparada com a abordagem em nuvem. Em rela\c{c}\~ao \`a performance do armazenamento
de dados, os resultados mostraram que a plataforma Fosstrak \'e adequada para processar os dados que
s\~ao produzidos num armaz\'em inteligente.\\

% Keywords
\keywords{Internet das Coisas, Computa\c{c}\~ao em Nevoeiro, Computa\c{c}\~ao em Nuvem, Deployment
de Aplica\c{c}\~oes, Plataforma Fosstrak}
\end{abstract}
\end{otherlanguage}
