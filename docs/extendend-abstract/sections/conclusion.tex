
%!TEX root = ../dissertation.tex

% Conclusion
\section{Conclusion and Future Work}
\label{sec:conclusion}
The present work explored the deployment of \gls{IoT} applications for smart warehouses based on the
\gls{RFID} technology with two different approaches to deploy: one based in a traditional cloud
deployment approach (cloud-based) and other according the Fog Computing platform (fog-based). More
specifically, the present work focuses to determine if a cloud-based approach is able to meet the
low-latency requirements of many \gls{IoT} applications, since that low-latency is an essential
requirement of \gls{IoT} applications. If a cloud-based approach is not able to meet the network latency
requirements for those applications, the cloud platform is not a viable option to perform the
provisioning of \gls{IoT} applications.\\

To improve the provisioning of \gls{RFID} middleware in the cloud, we developed a mechanism based on
Docker containers and the Chef tool that automates the installation and configuration of the modules
that composes the Fosstrak platform \gls{RFID} middleware. This mechanism was of extreme
importance, because it allowed us to perform the application provisioning of the cloud instances in
a very efficient way. Although our experiments were conducted in a single cloud provider, the developed
mechanism gave us the flexibility to choose between several cloud providers to provision the
\gls{RFID} middleware.\\

Regarding the system evaluation, we defined two methodologies for evaluate the latency of an event that
occurs in the physical space and the data storage performance for the Fosstrak platform. With the
methodologies proposed, we were able to compare the event latency performance for both cloud-based
and fog-based approaches. We defined two experiments to evaluate the latency performance of the
deployment approaches. The obtained results shows that the event latency performance presented better
results when the application was deployed according the fog-based approach. However, we identified
some issues regarding the behavior of a Fosstrak module (\gls{ALE}) that affected the performance
of the event latency for both deployment approaches. Regarding the data storage performance of the
RFID middleware, the results show that the Fosstrak platform is able to process with an acceptable
performance the amount of data that is generated in a smart warehouse.

% Future Work
\subsection{Future Work}
\label{sub:future_work}
In the present work, we achieved our initial goals and determine that a fog-based approach is more
adequate to deploy a smart place application based in \gls{RFID} technology. However, our solution
is not perfect and there some aspects that can be improved in the future.\\

Our solution proposes that the \gls{RFID} application is deployed following a fog-based approach.
This means that we need to have a cloud close to the ground and this cloud must meet the same
requirements of a remote cloud such as high scalability, security and multi-tenancy. Unfortunately,
we were not able to implement a fog that meet these requirements and in our implementation the fog
was built on top of a traditional Virtual Machine. In the future, the fog needs to be correctly
implemented providing all the features of the remote cloud and in addition features such as
location-awareness, mobility support and geo-distribution.\\

In the current implementation we used Docker containers to provisioning the Fosstrak software stack.
In the evaluation of our solution we deploy the containers in a \gls{EC2} \gls{VM}, which overlays two
different mechanisms of virtualization. Although we still are able to take advantage of some benefits
from the containers such as the portability, other benefits such as the low I/O and disk space are
hidden by the \gls{VM} hypervisor. A future improvement that can be made is to perform the deployment
of the containers on top of the bare-metal or in a cloud-based container service - e.g. Google Kubernetes
\footnote{\url{http://kubernetes.io/}} or \gls{AWS} \gls{EC2} Container Service\footnote{\url{https://aws.amazon.com/ecs/}} -
in order to improve the overall performance of the solution.\\

Regarding the system evaluation, it was performed only in \gls{AWS} \gls{EC2} instances. For the future
is important to evaluate our solution in other cloud providers to compare which offers the best cost/performance
relation. Also, in experiments performed to evaluate the latency interaction, we defined only two
different \textit{ECspecs}. For the future work, we want to evaluate the latency performance for
our solution with \textit{ECspecs} that presents smaller periods in order to determine which
specification is more suitable for our solution.\\

Finally, in the evaluation scenario we used a virtual RFID reader instead of a physical one, which
does not allow reproducing the environment conditions of a real smart warehouse such as interferences
in the RFID tags antennas, network bandwidth variations, etc. However, in the evaluation experiments
we used for some experiments traces from the work developed by Correia et. al \cite{Correia:Thesis:2014},
which have the real data traces mentioned above. A future improvement is to conduct the system evaluation
in a real scenario in order to have more accurate results.
