
%!TEX root = ../dissertation.tex

% Conclusion
\section{Conclusion and Future Work}
\label{sec:conclusion}
This dissertation explores the deployment of \gls{IoT} applications in the cloud and proposes
two different approaches to deploy an \gls{IoT} application for smart warehouses based on the
\gls{RFID} technology: one based in a traditional cloud deployment approach (cloud-based) and other
according the Fog Computing platform (fog-based). More specifically, the present work focuses to
determine if a cloud-based approach is able to meet the low-latency requirements of \gls{IoT}
applications. Since that low-latency is an essential requirement of \gls{IoT} applications, if a
cloud-based approach is not able to meet the network latency requirements for those applications,
the cloud platform is not a viable option to perform the provisioning of \gls{IoT} applications.\\

To improve the provisioning of \gls{RFID} middleware in the cloud, we developed a mechanism based on
Docker containers and the Chef tool that automates the installation and configuration of the modules
that composes the \gls{RFID} middleware, namely Fosstrak platform. This mechanism was of extreme
importance, since that allowed us to perform the application provisioning in the cloud instances in
a very efficient way. Although our experiments were conducted in a single cloud provider, the developed
mechanism give us the flexibility to chose between several cloud providers to provisioning the \gls{RFID} middleware.\\

Regarding the system evaluation, we defined a methodology for evaluate the latency of an event that
occurs in the physical space that allow us to compare the event latency performance for both cloud-based
and fog-based approaches. We defined two experiments to evaluate the latency performance of the
deployment approaches. The obtained results shows that the event latency performance presented better
results when the application was deployed according the fog-based approach. However, we identified
some issues regarding the behavior of a Fosstrak module (\gls{ALE}) that affected the performance
of the event latency for both deployment approaches.

% Future Work
\subsection{Future Work}
\label{sub:future_work}
In the present work, we achieved our initial goals and determine that a fog-based approach is more
adequate to deploy a smart place application based in \gls{RFID} technology. However, our solution
is not perfect and there some aspects that can be improved in the future.\\

Our solution proposes that the \gls{RFID} application is deployed following a fog-based approach.
This means that we need to have a cloud close to the ground and this cloud must meet the same
requirements of a remote cloud such as high scalability, security and multi-tenancy. Unfortunately,
we were not able to accomplish a fog that meet these requirements and in our implementation the fog
was built on top of a traditional Virtual Machine. In the future, the fog need be correctly
implemented providing all the features of the remote cloud and in addition features such as
location-awareness, mobility support and geo-distribution.\\

In the current implementation we used Docker containers to provisioning the Fosstrak software stack.
In the evaluation of our solution we deploy the containers in a \gls{EC2} \gls{VM}, which overlays two
different mechanisms of virtualization. Although we still are able to take advantage of some benefits
from the containers such as the portability, other benefits such as the low I/O and disk space are
hidden by the \gls{VM} hypervisor. A future improvement that can be made is to perform the deployment
of the containers on top of the bare-metal or in a cloud-based container service - e.g. Google Kubernetes
\footnote{\url{http://kubernetes.io/}}, \gls{AWS} \gls{EC2} Container Service\footnote{\url{https://aws.amazon.com/ecs/}} -
in order to improve the overall performance of the solution.\\

Regarding the system evaluation, it was performed only in \gls{AWS} \gls{EC2} instances. For the future
is important to evaluate our solution in other cloud providers to compare which offers the best cost/performance
relation. Also, in experiments performed to evaluate the latency interaction, we defined only two
different \textit{ECspecs}. For the future work, we want to evaluate the latency performance for
our solution with \textit{ECspecs} that presents smaller periods in order to determine which
specification is more suitable for our solution.
