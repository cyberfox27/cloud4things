
%!TEX root = ../dissertation.tex

% Introduction
\section{Introduction}
\label{sec:introduction}
The \gls{IoT} is an ubiquitous system composed of physical items or \textit{things} that are continuously
connected to the virtual world and can act as remotely and physical access points to Internet Services \cite{mattern2010internet}.
Many times these systems requires low-latency interaction with users and environments, which implies
that at least part of an \gls{IoT} application needs to be tightly bounded to the local infrastructure
of the interacting environment. This requirement for local infrastructure can be a barrier in the
adoption of those systems in a large-scale perspective. The Utility Computing in the cloud paradigms
can help to solve this issue by providing the illusion of infinite computing resources available on
demand to the public users \cite{armbrust2010view}. This paradigm can help to reduce the infrastructure
burden of \gls{IoT}, while providing important features such as high availability and high scalability.

% Application Domains
\subsection{Application Domains}
\label{subs:application_domains}
There are several environments and domains where \gls{IoT} applications are expected
to improve the quality of life of the people that lives and works in these environments.\\

% Ambient Intelligence
\textit{Ambient Intelligence} are environments that use the intelligence of the objects within itself to
become a more comfortable and efficient environment. For instance, an automated industrial plant
where its is possible to monitor the production progress.
% Logistics
In \textit{Logistics}, real-time information processing based on technologies such as \gls{RFID} allows to
achieve monitoring of almost every phase of the supply chain, ranging from raw material purchasing,
transportation and after-sales services.
% Transportation
A large area that is covered by \gls{IoT} is \textit{Transportation}. From personal vehicles to public transportation,
mobile ticketing and transportation of goods. Cars, trains and buses are equipped with sensors, actuators
and computational power that are able to provide information about the status of the vehicle, improve
the navigation and even perform collision avoidance.\\

To give a more unified view over those domains, we propose the term \textit{smart places},
that can be defined as an environment composed of sensors - e.g. RFID - actuators - e.g. automatic
doors - and computational infrastructure - e.g. servers - that are able to acquire data about the
surrounding environment and use that data to improve the experience of the people interacting with
the place.

% Smart Places Challenges
\subsection{Smart Places Challenges}
\label{substion:challenges}
Challenges \cite{caceres2012ubicomp} for the construction of smart places that resulted from leveraging
part of the smart place infrastructure to the cloud:\\

% Low-latency Interaction
\textit{a) Low-latency Interaction} is a key requirement that requires that both network access
latency and data transmission latency be reduced. However, will a cloud-based
approach able to meet the low-latency requirements of smart place applications?\\

% Data
\textit{b) Data} generated by the \textit{things} that composes the system must be stored, processed
and presented in a seamless and efficient way. Will the current middleware solutions for smart places
\cite{floerkemeier2007rfid}\cite{eisenhauer2010hydra}\cite{de2008socrades} be able to handle with
the volume of data generated by smart place applications?\\

% Deployment
\textit{c) Deployment} of smart places usually is performed in an isolated and vertical manner, which
makes the deployment its an inefficient process. Therefore, new provisioning approaches need to be
adopted in order to make the deployment smart place more efficient and scalable.\\

% Management
\textit{d) Management} of smart places is an issue that must be carefully addressed. Since
service providers will manage these services, which introduces new questions that must be
answered. For instance, who will pay and who will control these services?

% Objectives
\subsection{Objectives}
\label{subs:objectives}
The objective of this work is to determine if a cloud-based solution can meet the low-latency network
and data storage performance requirements for smart place applications. Our efforts will be engaged in
determine if a cloud-based solution is suitable for a smart place that relies on the RFID technology
\cite{want2006introduction}.\\

\subsubsection{Example Domain}
\label{subs:domain}
Our example domain is an automated warehouse, where products are tagged with \gls{RFID} tags that
can be identified by \gls{RFID} readers. Through the data collected by these readers is possible to
gather information about the smart place. A complete example of a platform that allows the transformation
of that data into information is Fosstrak\footnote{\url{http://fosstrak.github.io/}}, an open
source \gls{RFID} software that implements the \gls{EPC} Network standards.\\

The present work we will follow two approaches and determine which of them is more adequate to
deploy a \gls{RFID} application. The first, is an approach based on the Fog Computing \cite{bonomi2012fog},
a virtualized platform that is located close to the smart place and provides network, computing and storage
resources between the embedded devices in the physical place and the traditional cloud. The second is
a traditional approach.\\

The remainder of this document is organized as follows. Section \ref{sec:background} summarizes the
relevant work in the field and introduces the base concepts of our work such as a description of the
Fog Computing paradigm and the Fosstrak platform. In Section \ref{sec:solution} we present the approaches
adopted to deploy the smart place software stack. Section \ref{sec:implementation} details the
implementation for the proposed solution. In Section \ref{sec:evaluation} describes the performed
evaluation and presents an analysis of the obtained results and finally in Section \ref{sec:conclusion}
we present the main conclusion and some important research points for future work.
